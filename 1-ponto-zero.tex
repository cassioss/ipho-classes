% !TEX encoding = UTF-8 Unicode
% !TEX encoding = UTF-8 Unicode
\documentclass[12pt, a4paper]{article}
\pagestyle{headings}

\usepackage[brazil]{babel}
\usepackage[utf8]{inputenc}
\usepackage[T1]{fontenc}
\usepackage{textcomp}

\usepackage{amssymb, amsmath, pxfonts}

\usepackage{verbatim}
\usepackage{graphicx}
\usepackage{latexsym}
\usepackage{mathrsfs}

\usepackage{indentfirst}										% Adds tabs automatically

\usepackage[normalem]{ulem}										% Underlines words
\usepackage[top=3cm,left=3cm,right=3cm,bottom=3cm]{geometry}	% Margins

\usepackage[usenames]{color}									% Colored letters
\usepackage{makeidx}											% To create the index

\expandafter\def\expandafter\normalsize\expandafter{%
    \normalsize
    \setlength\abovedisplayskip{2pt}
    \setlength\belowdisplayskip{12pt}
    \setlength\abovedisplayshortskip{2pt}
    \setlength\belowdisplayshortskip{12pt}
}

\newcommand{\ui}{\mathrm{i}}
\newcommand{\ud}{\mathrm{d}}

\author{Cássio dos Santos Sousa}
\title{Energia do Ponto Zero}

\begin{document}
\maketitle

\section{Introdução}

Todos os estados energéticos de uma partícula podem ser deduzidos a partir da \textbf{equação de Planck}:

\begin{equation}
    \epsilon = \frac{h\omega}{e^{\frac{h\omega}{kT}} - 1}
    \label{eq1}
\end{equation}

No entanto, o que se percebe é que, no chamado \textbf{ponto zero}, onde uma partícula não se encontra deslocada da origem nem se movimentando, ela possui sim energia. Isso concorda com a \textbf{Equação de Schr\"odinger}, que diz que toda partícula possui uma incerteza mínima em sua posição e velocidade:

\begin{equation}
    \Delta p \Delta x \geq \frac{\hbar}{2} = \frac{h}{4\pi}
    \label{eq2}
\end{equation}

Vamos então descobrir qual é o valor dessa energia a partir da Equação de Schr\"odinger.

\section{Prova utilizando o Princípio da Incerteza}

A posição e o momento linear de uma partícula são dados por meio de um valor médio e de uma incerteza:

\begin{eqnarray}
    p = \bar{p} + \Delta p  \label{eq3} \\
    x = \bar{x} + \Delta x  \label{eq4}
\end{eqnarray}

Se, pela definição do ponto zero, a partícula não se encontra deslocada da origem nem se movimentando, então, no caso médio:

\begin{equation}
    \bar{p} = \bar{x} = 0   \label{eq5}
\end{equation}

Logo, no ponto zero:

\begin{eqnarray}
    p = \Delta p  \label{eq6} \\
    x = \Delta x  \label{eq7}
\end{eqnarray}

Do operador Hamiltoniano, temos que a energia de uma partícula de massa $m$ com uma frequência natural ${\omega}_0$ é dada por:

\begin{equation}
    H = V_0 + \frac{k x^2}{2} + \frac{p^2}{2m}
    \label{eq8}
\end{equation}

E nesta equação:

\begin{equation}
    k = m {{\omega}_0}^2
    \label{eq9}
\end{equation}

Por conta das equações (\ref{eq6}) e (\ref{eq7}), teremos:

\begin{equation}
    H = V_0 + \frac{k {\Delta x}^2}{2} + \frac{{\Delta p}^2}{2m}
    \label{eq10}
\end{equation}

No entanto, pelo Princípio da Incerteza descrito em (\ref{eq2}), temos que:

\begin{equation}
    \Delta p \geq \frac{\hbar}{2 \Delta x}
    \label{eq11}
\end{equation}

Utilizando (\ref{eq11}) em (\ref{eq10}):

\begin{equation}
    H \geq V_0 + \frac{k {\Delta x}^2}{2} + \frac{{\hbar}^2}{8m{\Delta x}^2}
    \label{eq12}
\end{equation}

É sempre possível derivar esta expressão e encontrar seu valor mínimo. No entanto, vamos reescrevê-la um pouco antes:

\begin{equation}
    H \geq V_0 + \frac{\hbar}{4}\sqrt{\frac{k}{m}} \left( \frac{2{\Delta x}^2 \sqrt{km}}{\hbar} + \frac{\hbar}{2{\Delta x}^2 \sqrt{km}} \right)
    \label{eq13}
\end{equation}

Nesta equação, vemos que $H$ é maior ou igual à soma de um número com seu inverso. No entanto, sabemos que o quadrado de qualquer número real é maior ou igual a zero. Logo, se utilizarmos $w > 0$:

\begin{equation}
    \left(\sqrt{w} - \sqrt{\frac{1}{w}} \right)^2 \geq 0   \label{eq14}
\end{equation}

Desenvolvendo esta equação, podemos obter que:

\begin{equation}
    w + \frac{1}{w} \geq 2   \label{eq15}
\end{equation}

Ou seja, se $w > 0$, a soma de $w$ com seu inverso será sempre maior ou igual a 2. Logo, tomando:

\begin{equation}
    w = \frac{2{\Delta x}^2 \sqrt{km}}{\hbar}   \label{eq16}
\end{equation}

E utilizando a eq. (\ref{eq15}) na eq. (\ref{eq13}), temos que:

\begin{equation}
    H \geq V_0 + \frac{\hbar}{2}\sqrt{\frac{k}{m}}   \label{eq17}
\end{equation}

Utilizando agora a relação da eq. (\ref{eq9}), chegamos que:

\begin{equation}
    H \geq V_0 + \frac{\hbar {\omega}_0}{2}   \label{eq18}
\end{equation}

O termo $\frac{\hbar {\omega}_0}{2}$ é chamado de \textbf{energia do ponto zero}, e pode ser interpretada como a energia advinda do próprio fato de existir incertezas na posição e no momento, justificando-as.

\end{document}